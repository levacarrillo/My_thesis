\chapter{Discusión}
\section{Conclusiones}
	En este trabajo se fueron integrando conocimientos adquidridos de diversas ramas del conocimiento y materias que se imparten desde los inicios de la carrera de Ingeniería en Mecatrónica, tales como Cálculo Diferencial e Integral, Geometría Analítica, Álgebra Lineal, Cinemática y Dinámica, Ecuaciones Diferenciales, Estadística, Fundamentos de Programación, Programación Orientada a Objetos, Control Automático, Robótica y sobre todo Construcción de Robots Móviles.	
	
	Con respecto a la parte de resultados, se puede concluir que los objetivos planteados en esta tesis se cumplieron satisfactoriamente en cuanto al desarrollo del sistema de visión del humanoide (tanto el real como el simulado), el algoritmo de pateo con un control a lazo abierto basado en posiciones predefinidas y en la aplicación del Filtro de Kalman Extendido para la filtración del ruido inherente a las mediciones.

	Al momento de unir los distintos programas en el sistema embebido se presentaron algunos retos a cumplir en cuanto a la optimización de los procesos y su debida interconexión, para lo cual las herramientas de la plataforma ROS fueron de vital importancia y ayudaron para integrar nodos escritos en diferentes lenguajes de programación.

	Se logró crear la clase $Humanoid$ cuya funcionalidad es hacer que el robot se mueva a distintas posiciones, utilizando archivos de configuración con formato $.yaml$, que contiene los ángulos de cada actuador para cada movimiento. Para esta tesis solamente se utilizó una serie de movimientos perfilizados para el pateo. No obstante es fácil agregar más archivos $.yaml$ para crear una infinidad de posiciones haciendo que el humanoide tenga un mayor grado de biomímesis.

	En cuanto al objetivo general, se logró cumplir que el robot simulado pateara el balón con una previa predicción de posición. No obstante, por las dificultades causadas por la pandemia, quedará en espera el hacer pruebas con el robot real, con la expectativa de que se requieran sólo cambios mínimos concernientes al hardware y al ruido de medición con variables reales.
	
	Una de las ventajas de hacer las pruebas del robot simulado, que al mismo tiempo es una desventaja para el robot real, es el del tiempo de latencia que hay entre la comunicación con la computadora con el hardware del humanoide, es importante de mencionar debido a que representaría un retardo de tiempo a desde que se manda un mensaje hasta que el hardware lo reciba. 
	

	
\section{Trabajo Futuro}
	Como se vio en la parte de resultados el rango de velocidades del balón tuvo un umbral de 2.0 a 2.8 m/s aproximadamente, por lo que puede decirse que es un rango estrecho, y cuando se implemente en el mundo real puede disminuir dependiendo las variaciones del coeficiente de fricción. Una manera en que se mejoraría lo anterior, es adaptando una cámara con mayor rango de visión horizontalmente con las ya conocidas cámaras de \textit{ojo de pezcado}, así se podrían tomar las ocho muestras de información para que el estimador pueda hacer una estimación (valga la redundancia) con mayor tiempo y distancia. Para hacer más robusto el sistema de segmentación de objetos (ya que con la actual puede verse afectada por las tonalidades de luz que hay en el ambiente), alguna de las formas que sería conveniente probar es con la ayuda de una cámara estereoscópica o con una equipada con sensores láser a fin de obtener la profundidad con una nube de puntos.

	Para mejorar algoritmo de pateo en cuanto a su estabilidad en cada movimiento, sería conveniente agregar un sistema de control de velocidad (ya que sólo cuenta control de posición en cada articulación). A parte de esto sería muy conveniente que se hiciera un control de posiciones con realimentación, ya que es fácil que el robot caiga con cada prueba. Con un sistema a lazo cerrado se obtendría la estabilidad necesaria e incluso se podrían realizar movimientos más rápidos, ampliando sin duda el rango de velocidades a los que el robot pueda patear un balón en movimiento. 
		
	Podría decirse que en este trabajo se desarrolló un sistema de visión de predicción de posición análogo al del ser humano, utilizando un método basado en la esperanza de una serie de datos y un previo modelo matemático. Sin embargo, esta no es la única forma de emular este tipo de funciones, para futuras propuestas de proyectos, se podría implementar un sistema de estimación basado el redes neuronales o algoritmos genéticos, los cuales no tendrían las mismas limitantes que el Filtro de Kalman, como por ejemplo que se tenga que conocer el coeficiente de fricción dinámica, que se requiera una superficie completamente plana u horizontal y que el balón no tenga rebotes o elevaciones de ningún tipo en cuanto a su trayectoria.

\appendix
\section{Apéndice A: Código del estimador de Kalman}
\lstinputlisting[language=Python]{kalman_estimator.py}