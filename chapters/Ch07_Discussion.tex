\chapter{Discusión}
\section{Conclusiones}

	Con respecto a la parte de resultados, se concluye que los objetivos planteados en esta tesis se cumplieron satisfactoriamente en cuanto al desarrollo del sistema de visión del humanoide (tanto el real como el simulado), el algoritmo de pateo con un control a lazo abierto basado en posiciones predefinidas y en la aplicación del Filtro de Kalman Extendido para la filtración del ruido inherente a las mediciones.

	Al momento de unir los distintos programas en el sistema simulado se presentaron algunos retos a cumplir en cuanto a la optimización de los procesos y su debida interconexión, para lo cual las herramientas de la plataforma ROS fueron de vital importancia y ayudaron para integrar nodos escritos en diferentes lenguajes de programación.

	Se logró crear la clase $Humanoid$ cuya funcionalidad es hacer que el robot se mueva a distintas posiciones, utilizando archivos de configuración con formato $.yaml$, que contiene los ángulos de cada actuador para cada movimiento. Para esta tesis solamente se utilizó una serie de movimientos perfilizados para el pateo. No obstante es fácil agregar más archivos $.yaml$ para crear una infinidad de posiciones haciendo que el humanoide tenga un mayor grado de biomímesis.

	En cuanto al objetivo general, se logró cumplir que el robot simulado pateara el balón con una previa predicción de posición. Sin embargo, quedará en espera el hacer pruebas con el robot real, con la expectativa de que se requieran sólo cambios mínimos concernientes al hardware y al ruido de medición con variables reales.
	
	Una de las ventajas de hacer las pruebas del robot simulado, que al mismo tiempo es una desventaja para el robot real, es el del tiempo de latencia que hay entre la comunicación con la computadora con el hardware del humanoide, es importante de mencionar debido a que representaría un retardo de tiempo a desde que se manda un mensaje hasta que el hardware lo reciba. 
	

	
\section{Trabajo Futuro}

	Como trabajo futuro, se deja como prioritario migrar todos los programas y parámetros al sistema embebido, para así dejarle completamente listo para las competencias internacionales y para el seguimiento del desarrollo de software que otros alumnos de la Facultad de Ingeniería estén dispuestos a hacer.

	 En la sección 4.2 el modelo matemático se obtuvo como una deducción basada en la segunda ley de Newton y la posición del balón está descrita de manera parabólica respecto al tiempo y aunque, es intuitivamente viable, sería útil implementar un modelo matemático distinto con una descripción hiperbólica, igual que la ecuación de la fuerza de fricción viscosa.
	 
	Tal y como se vio en la parte de resultados el rango de velocidades del balón tuvo un umbral de 2.0 a 2.8 m/s aproximadamente, es decir, que es un rango estrecho, y cuando se implemente en el mundo real puede disminuir dependiendo las variaciones del coeficiente de fricción. Una manera en que se mejoraría lo anterior, es adaptando una cámara con mayor rango de visión horizontalmente con las ya conocidas cámaras de \textit{ojo de pezcado}, así se podrían tomar las ocho muestras de información para que el estimador pueda hacer una estimación (valga la redundancia) con mayor tiempo y distancia. Para hacer más robusto el sistema de segmentación de objetos (ya que con la actual puede verse afectada por las tonalidades de luz que hay en el ambiente), una forma conveniente para probar, es con la ayuda de una cámara estereoscópica a fin de obtener la profundidad con una nube de puntos.

	Para mejorar algoritmo de pateo en cuanto a su estabilidad en cada movimiento, es conveniente agregar un sistema de control de velocidad (ya que sólo cuenta control de posición en cada articulación). A parte de esto, que se hiciera un control de posiciones con realimentación, ya que es fácil que el robot caiga con cada prueba. Con un sistema a lazo cerrado se obtendría la estabilidad necesaria e incluso se podrían realizar movimientos más rápidos, ampliando sin duda el rango de velocidades a los que el robot pueda patear un balón en movimiento. 
		
	En este trabajo se desarrolló un sistema de visión de predicción utilizando un método basado en la esperanza de una serie de datos y un previo modelo matemático. Sin embargo, esta no es la única forma de hacer este tipo de funciones, para futuras propuestas de proyectos, se podría implementar un sistema de estimación basado el redes neuronales (más parecido al sistema del ser humano) o algoritmos genéticos, los cuales no tendrían las mismas limitantes que el Filtro de Kalman, como por ejemplo que se tenga que conocer el coeficiente de fricción dinámica, que se requiera una superficie completamente plana u horizontal y que el balón no tenga rebotes o elevaciones de ningún tipo en cuanto a su trayectoria.
	
	