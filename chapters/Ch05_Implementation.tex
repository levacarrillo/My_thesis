\chapter{Implementación}
	\section{Las bilbiotecas OpenCV}
OpenCV (\textit{Open Computer Vision}) es una biblioteca \textit{open source} diseñada para visión computacional. Esta biblioteca está escrita principalmente en C y C$++$ y es capaz de ejecutarse en diversos sistemas operativos, tales como: \textit{Windows}, \textit{Linux}, o \textit{Mac OS X}.

OpenCV fue desarrollada para hacer procesos más eficientes cuando se hace visión con aplicaciones en tiempo real. Una de las metas de OpenCV es proveer una infraestructura sencilla y sofisticada para  diferentes tipos de usuarios, que van desde profesores, estudiantes, profesionistas, desarrolladores, y autodidactas. La librería cuenta con más de 500 funciones que se extienden en diversas áreas de visión, incluyendo inspección en la fabricación de productos, seguridad, calibración de cámaras, visión estéreo, robótica, etc.

	\section{La plataforma ROS}
		\subsection*{¿Qué es ROS?}
Citando a \cite{pyo2015ros} ROS es un meta sistema operativo \textit{open-source} que provee servicios a las aplicaciones de robótica, servicios que comunmente se esperan de un sistema operativo, tales como: abstracción de hardware, control de dispositivos a \textit{bajo nivel}, paso de mensajes entre procesos, ordenamiento y manejo de distintos tipos de paquetes. ROS también provee herramientas y bibliotecas para obtener, construir, escribir y ejecutar programas a través de multiples computadoras.\\

ROS es la abreviación en inglés de \textit{Robot Operating System} lo cual se prodría traducir al español como Sistema Operativo de Robots. Con este nombre se podría pensar que ROS es un sistema operativo, sin embargo el término mejor empleado es el de \textit{Meta Sistema Operativo}. Aunque \textit{Meta Sistema Operativo} no está definido en el diccionario, se puede describir como un sistema que realiza procesos tales como programación, ejecución, monitoreo, y manejo de errores, utilizando una capa de visualización entre aplicaciones y recursos informáticos distribuidos.\\

Dicho lo anterior, ROS no es un sistema operativo convencional, tal como \textit{Windows}, \textit{Linux}, o \textit{Android}, sino una plataforma que se ejecuta dentro del sistema operativo instalado. A menudo, para utilizar ROS se requiere tener instalado \textit{Ubuntu}, que es un sitema basado en las distribuciones de \textit{Linux}. No obstante, es posible usarse en distintos sitemas, tal y como se muestra la figura \ref{fig:meta_operating_system}. 
\begin{figure}
\centering
\includegraphics[scale=0.5]{images/meta_operating_system.png}
\caption{Esquema de la abstracción de harware por ROS en distintos Systemas Operativos}
\label{fig:meta_operating_system}
\end{figure} 

		\subsection*{Objetivos al utilizar ROS}
Aunque existen diversas plataformas de robótica (OpenRTM, OPRoS, Player, Orca, Microsoft Robotics Studio, etc.), ROS está orientado a construir entornos de desarrollo para software de robótica a un nivel global, con esto se espera que el código de diferentes desarrolladores se pueda usar, modificar o mejorar para hacer crecer el entorno mismo. Para hacer de una manera sencilla el entorno de desarrollo, ROS tiene las siguientes características:

\begin{itemize}
\item \textbf{Distribución de procesos:}
Están programados en unidades mínimas de procesamiento (nodos). Cada uno de estos procesos se ejecuta de manera independiente y es capaz de intercambiar datos con otros de manera sistemática.  
\item \textbf{Manejo de paqueterías:}
Cuando varios procesos tienen propósitos similares, estos se manejan dentro de un \textit{paquete} que haga los procesos más ordenados y fáciles de desarrollar.
\item \textbf{Repositorios públicos:}
Cada paquete se hace público dentro de un repositorio (por ejemplo GitHub) para que la comunidad de desarrolladores puedan acceder a él. 
\item \textbf{API (Interfaz de Programación de Aplicaciones):}
Cuando se desarrolla un programa en ROS, generalmente se llaman funciones ya existentes fácilmente insertarlas dentro del código que se esté construyendo.
\item \textbf{Soporte de distintos lenguajes de programación:}
La plataforma ROS posee una \textit{bilioteca de clientes} para facilitar el trabajo de los programadores. La biblioteca puede importar lenguajes de programación que son bastantes populares, tales como Python, C$++$, Java, Ruby, Lips, entre otros.
\end{itemize}
		\subsection*{Breve historia de ROS}
\textit{Robot Operating System} fue creado en Mayo del 2007 por el Doctor Morgan Quigley (véase la figura \ref{fig:morga_quigley}) dentro del \textit{Standford Artificial Intelligence Laboratory}. Se podría decir que el predecesor de ROS es un proyecto llamado \textit{SwitchYard}, el cual es un programa creado para el desarrollo de inteligencia artificial de robots.

En noviembre del 2007 la compañía estadounidense \textit{Willow Garage} empezó el desarrollo de ROS dentro del campo de los robots de servicio. De esta manera ROS vino al mundo oficialmente el 22 de Enero del 2010 con la versión llamada \textit{ROS 1.0}, sin embargo la versión más conocida fue \textit{Box Turtle} lanzada en Marzo del mismo año.

La plataforma ROS es actualizada cada dos años (entre el lapso de Abril-Octubre), lo que significa que cada versión tiene soporte durante la misma cantidad de tiempo. Para propósitos de esta tesis se optó por utilizar la versión \textit{ROS-Kinetic} instalado en Ubuntu 16.04 \textit{Xenial Xerus (LTS)}.
\begin{figure}
\centering
\includegraphics[scale=0.7]{images/morgan_quigley.jpg}
\caption{Dr. Morgan Quigley, uno de los principales fundadores de ROS. Imagen tomada de:}
\label{fig:morga_quigley}
\end{figure}
