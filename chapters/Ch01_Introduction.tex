\chapter{Introducción}

\section{Motivación}
	\textbf{El robot humanoide tipo \textit{Nimbro-OP}, forma parte del laboratorio de Bio-robótica de la Facultad de Ingeniería de la UNAM, tiene sistemas de visión computacional para detección y localización de objetos estáticos, algoritmos de caminado y pateo, entre otras características; Este es un proyecto en desarrollo que requiere tener la habilidad de interceptar objetos en movimiento, como un balón de fútbol}
\section{Planteamiento del problema}
\section{Hipótesis}
\textcolor{red}{Agrega un par más que vayan acordes con los objetivos.}
		\textbf{Mediante el Filtro de Kalman se puede predecir la posición de un objeto en movimiento para ser iterceptado por el actuador de un robot}
\section{Objetivos}
	\subsubsection{General}
		\begin{itemize}
			\item \textbf{Aplicar un Filtro de Kalman para la predicción de posición y velocidad de un objeto en movimiento para un robot bípedo tipo Nimbro-OP}
		\end{itemize}
	\subsubsection{Particulares}
	\textcolor{red}{Estos hay que reescribirlos.}
		\begin{itemize}
			\item \textbf{Uso de las librerías de OpenCV de procesamiento de imágenes}
			\item \textbf{Utilización de las herramientas de la plataforma ROS para localización de un objeto de interés en el espacio}
			\item \textbf{Integración de programas en un sistema embebido para el pateo de un balón en movimento}
		\end{itemize}		
\section{Descripción del documento}