\chapter{Antecedentes}
	\section{Robots bípedos}
		\subsection*{Conceptos básicos}
Un robot con piernas es un robot móvil que debe tener un cuerpo, al menos una pierna (extremidad inferior) y un número arbitrario de brazos (extremidades superiores). Generalmente sus piernas deben tener un actuador final con el cual apoyarse e impusarse  y sus brazos, un actuador para manipular objetos \citep{siciliano2016springer}. Por lo tanto, se puede deducir que un \textit{robot bípedo} es un robot con dos piernas las cuales usa para moverse, de forma similar al caminado.
\\

Entre los \textit{robots bípedos} más conocidos se encuentran los \textit{robots humanoides} (figura \ref{fig:humanoids}) que poseen las siguientes características[]:

\begin{itemize}
\item Tienen la apariencia y forma de un ser humano, por lo que su cuerpo debe consistir de dos brazos, dos piernas y una cabeza ajustada a un tronco.
\item Deber ser capaces de permanecer de pie sobre sus pies y caminar con sus piernas.
\item Pueden interactuar con humanos usando reconocimiento de voz y/o de imágenes.
\item Los movimientos que son capaces de hacer deben ser cinenmáticamente equivalentes a los de un ser humano por ejemplo, las articulaciones de la rodilla no deben doblarse para atrás, la cabeza no debe girar a más de 180 grados, ni debe tener actuadores lineales en sus extremidades.
\end{itemize}

Uno de los mayores retos a la hora de diseñar y modelar a los robot bípedos es su movimiento. A comparación de otros robots móviles, los robots con piernas tienden a tener un mayor número de grados de libertad en sus extremidades y estos deben estar muy bien coordinados para no hacer caer al robot. 
\\
			\subsubsection*{Características para el diseño de un robot con piernas}
\begin{itemize}
\item \textbf{Tipo de marcha:} Es el patrón de movimientos de piernas del robot (caminata).

\item \textbf{Biomímesis:} Es el diseño de algunos robots para imitar la estructura mecánica de un ser vivo de tal manera que sea tan precisa como se pueda.

\item \textbf{Bioinspiración mecánica:} Es el diseño que sirve para reproducir la robustez y versatilidad de la locomoción de animales, algunos diseñadores prestan más atención en la dinámica esencial de la locomoción que en la mecánica.

\item \textbf{Simplicidad mecánica:} Con esto se pretende usar el menor número de actuadores posibles para cumplir sólo con las tareas realizadas.

\item \textbf{Espacio de trabajo de la extremidad: } Señala que una extremidad debería tener al menos 3GDL para moverse libremente en el espacio. Para que se pueda tener una arbitraria orientación en el efector final en un espacio 3-D se debe contar con almenos 6GDL.

\end{itemize}

\begin{figure}
	\centering		
	\includegraphics[scale=0.5]{images/asimov_and_HRP-4C.png}
	\caption{Ejemplos de robots humanoides (a) Asimov (2000); (b)HRP-4C (2009) (Tomado de: Siciliano and Khatib, 2016).}		
\label{fig:humanoids}%(Ver página 423).
\end{figure}


		\subsection*{Modelado y Control de Robots con Piernas}
			\subsubsection*{Breve Historia de los Robots con piernas *Pendiente}
		Los robots con piernas controlados digitalmente empezaron a aparecer a finales de los 1960's. Entre los pioneros, Robert McGhee inicializó una serie de robots cuadrúpedos y hexápodos en \textit{University of South California}.\\

			\subsubsection*{Dinámica del movimiento de los sistemas con piernas}
Como ya se había mencionado, las mayores dificultades para hacer que un robot camine o corra es mantener su balance, de esta forma suelen formularse las siguientes preguntas: ¿Dónde debería caminar el robot?, ¿Cómo debería moverse su cuerpo a fin de moverse de manera segura en una dada dirección, incluso si hay fuertes perturbaciones?.\\ 

			\subsubsection*{Análisis de estabilidad}

Para el control del sistema dinámico no lineal hay  un número de conceptos relativos para su seguridad y estabilidad:

\begin{itemize}
\item \textit{Puntos fijos} : Representan las posturas estáticas en cuáles el robot puede estar de pie de manera segura.

\item \textit{Ciclos límites}: Proveen una natural extensión del análisis de los puntos fijos para movimientos de caminata periódica.

\item \textit{Viabilidad}: La viabilidad es un concepto de invarianza controlada, que analiza el conjunto de estados del cual el robot es capaz de mantenerse de pie. Desafortunadamente esta propiedad puede ser intratable para el cómputo.

\item \textit{Controlabilidad}: La controlabilidad provee una ligera noción de restricción de viabilidad, analizando el conjunto de estados del cual el robot es capaz de retornar a un particular punto fijo.

\item \textit{Estabilidad robusta}: Examina las propiedades del sistema considerando el "peor de los casos" de las perturbaciones. Por instancias, un controlador robusto debería ser capaz de garantizar que un punto fijo es estable incluso si la estimación de masa del tronco tiene un error del $\pm$10\%.

\item \textit{Estabilidad estocástica}: El análisis estocástico provee herramientas para investigar la probabilidad de caer. Para muchos modelos de perturbaciones en robots el sistema caerá eventualmente (con probabilidad uno), pero el análisis puede revelar la distribución del tiempo de vida metaestable.

\item \textit{Estabilidad de entrada-salida}: En este análisis se trata una particular perturbación como una entrada, un criterio de rendimiento como salida e intenta calcular una ganancia relativa o sensibilidad del rendimiento del robot debido a esta entrada.

\item \textit{Márgenes de estabilidad}: El análisis de robustez puede ser difícil. En la práctica, los diseñadores del control a menudo se conforman con que el sistema se mantenga cómodamente lejos de los límites de estabilidad determinista.		

\end{itemize}

%	\section{Aplicaciones en el juego de Futbol Soccer}
%	\section{Conceptos básicos de visión computacional}
%	\section{El filtro de Kalman}
%	\section{Trabajo relacionado}		


	\section{Conceptos básicos de visión computacional}
		\subsection*{Visión computacional}
	La \textit{visión computacional} es la transformación de datos desde una imagen o video cámara hacia una \textit{nueva representación} para lograr un fin en particular \cite{bradski2008learning}. Al decir esto, significa que se pueden usar ciertas características de la luz capturada en una imagen (o secuencia de imágenes) para transformarlas en variables numéricas que la computadora pueda abstraer y procesar (tómese de ejemplo la figura \ref{fig:camera_representation}).

\begin{figure}
\centering
\includegraphics[scale=0.6]{images/new_representation_image.png}
\caption{Representación de cómo una imagen es representada por la computadora. Imagen modificada de:}
\label{fig:camera_representation}
\end{figure}

Para los fines de este trabajo solamente se utilizarán funciones para segmentar objetos con base en su color, para obtener la posición del objeto en el espacio se hizo el análisis matemático de la sección 4.1.
	
	\section{El Filtro de Kalman}
		\subsection*{Filtro de Kalman Discreto}
	En 1060 Rudolf Emil Kalman publicó su famoso \textit{paper} en donde describe una recursiva solución para el problema del filtrado de datos discretos. Desde ese entonces el filtro ha sido sujeto de extensas investigaciones y aplicaciones en la ingeniería.

	El Filtro de Kalman es un conjunto de ecuaciones matemáticas que proveen un eficiente método computacional para estimar el estado de un proceso, de modo que minimiza el error producido por el ruido. Este filtro es una poderosa herramienta en varios aspectos tales como la estimación del pasado, presente y futuro, incluso es capaz de precisar el modelo de un sistema cuya naturaleza es desconocida.
	
		\subsection*{Proceso de estimación}
		El Filtro de Kalman aborda el problema de tratar de estimar el estado $x \in \Re^n$ de un proceso de tiempo discreto que es modelado con la ecuación diferencial \ref{eq:kalman_filter}.
		
\begin{equation}
x_k = Ax_{k-1} + Bu_{k-1}+w_{k-1}
\label{eq:kalman_filter}
\end{equation}

En donde $A$ es una matriz de nxn que relaciona el estado anterior $k-1$ con el estado presente $k$. $B$ es una matriz de $nx1$ que relaciona la entrada de control opcional $u \in R^1$ al estado $x$. 
La medición $z \in \Re^m$, también conocida como matriz de observación se puede modelar con la ecuación \ref{eq:measurement_equation}:

\begin{equation}
z_k = Hx_k+v_k
\label{eq:measurement_equation}
\end{equation}

En \ref{eq:measurement_equation} $H$ es una matriz de $mxn$ que relaciona el estado con la medición $z_k$.  Las variables aleatorias $w_k$ y $v_k$ representan el ruido del proceso y de la medición (respectivamente), asumiendo que son independientes una de la otra. Las expresiones en \ref{eq:normal_distribution_w} y \ref{eq:normal_distribution_R} representan su distribucion normal.

\begin{equation}
p(w) \sim N(0,Q),
\label{eq:normal_distribution_w}
\end{equation}
\begin{equation}
p(w) \sim N(0,R)
\label{eq:normal_distribution_R}
\end{equation}

Con el objetivo de encontrar una ecuación que calcule una estimación \textit{a posteriori} de un estado $\hat{x}_k$ como una combinación lineal de un estado \textit{a priori} $\hat{x}_k^-$ y una proporcional diferencia de una medición actual $z_k$ con una medición de la predicción $H\hat{x}_k^-$ se emplea la ecuación \ref{eq:posteriori_predicted}.

\begin{equation}
\hat{x}_k = \hat{x}_k^- + K(z_k-H\hat{x}_k^-)
\label{eq:posteriori_predicted}
\end{equation}

La diferencia $(z_k-H\hat{x}_k^-)$ dentro de la ecuación \ref{eq:posteriori_predicted} es llamada \textit{innovación} o \textit{residuo}. $K$ es una matriz de $nxm$ que representa la ganancia que reduce el error de covarianza \textit{a posteriori}. Esta ganancia está dada por la ecuación \ref{eq:k_gain}.

\begin{equation}
K_k = P_k^-H^T(HP_k^-H^T+R)^{-1} = \frac{P_k^-H^T}{HP_k^-H^T+R}
\label{eq:k_gain}
\end{equation}

De esta manera, cuando el error de covarianza en la medición $R$ se aproxima a cero, la ganancia $K$  tiene un mayor peso en la ecuación. Por el contrario, cuando el error de covarianza \textit{a posteriori} $P_k^-$ se aproxima a cero, la ganancia $K$ obtiene menor peso. 

Otra forma de entender la utilidad de la ganancia de $K$ es pensar que mientras la ganancia de aproxime a 1, hará que la medición sea más \textit{confiable}, cuando las mediciones se aproximen a cero, serán menos confiables y se tomará como confiable los datos relacionados con el modelo matemático.
		\subsubsection{Algoritmo de filtrado}
El algoritmo consta de dos fases, la fase \textit{predictiva} y la fase \textit{correctiva}. La fase predictiva consta de utilizar un modelo matemático para obtener las posibles mediciones de un fenómeno. La fase correctiva obtiene los valores de la medición y las compara con las obtenidas con el modelo matemático, de manera que se obtienen los valores de salida, tal y como se observa en la figura \ref{eq:kalman_filter}.

\begin{figure}
\centering
\includegraphics[scale=0.7]{images/kalman_algorithm.pdf}
\caption{Esquema ilustrativo del Filtro de Kalman}
\label{fig:Kalman_scheme}
\end{figure}
