\chapter{Estimación de posición y velocidad}
	\section{Cálculo de la posición del objeto de interés}
Tal como se observa en la simulación de Gazebo (Figura 3.1), se decidió usar un sistema ortogonal derecho en la base de los pies como sistema de referencia que gobernará a todo el modelo.

\begin{figure}
	\centering		
	\includegraphics[scale=2]{images/robot_ejes.png}
	\caption{Representación del sistema de referencia usado en el robot.}		
\end{figure}

La cámara está localizada en la cabeza del humanoide, por lo que su centro
de visión se puede representar por un eje que va de la cámara al centroide del 
objeto, en este caso un balón de fútbol.
Para estimar la posición de un objeto que cruce por el centro de visión de la
cámara se necesita establecer un vector unitario:
\[\hat{u} = (u_x, u_y, u_z)\]
conocido en computación gráfica como vector \textit{look at}, (ver figura \ref{fig:LookAt}). Para obtener la ecuación vectorial de la recta paralela al vector \textit{look at} se toma un punto que contenga la recta, en este caso la posición de la cabeza en donde se encuentra la cámara: 

\begin{figure}
	\centering		
	\includegraphics[scale=2]{images/robot_lookat.png}
	\caption{Representación del vector \textit{look at} utilizado en visión computacional.}		
	\label{fig:LookAt}
\end{figure}

\begin{equation}
\label{eq:LookAt}
r(\lambda) = (r_x, r_y, r_z)\quad +\quad \lambda (u_x, u_y, u_z)
\end{equation}


En donde $(r_x, r_y, r_z)$ es la posición en el espacio de la cámara utilizada, referida al sistema de referencia anteriormente mencionado. Se considera que el objetivo siempre estará en el suelo, por lo que el punto de intersección de la ecuación de la recta con el suelo hacen que la variable de altura sea igual a cero, conforme a la siguiente expresión:
\[r = (x, y, 0) = (r_x, r_y, r_z) + \lambda (u_x, u_y, u_z)\]
Despejando $\lambda$ del tercer término de la expresión anterior se obtiene:
\[\lambda = -\frac{r_z}{u_z}\]

De esta manera, substituyendo $\lambda$ en (\ref{eq:LookAt}), se obtiene:
\[x = r_x-\frac{r_z u_x}{u_z}\]
\[y = r_y-\frac{r_z u_y}{u_z}\]

Ya teniendo estas expresiones se procede a sustituir al vector unitario \textit{look at} con coordenadas esféricas, tal y como se observa en la siguiente expresión:
\begin{equation}
\label{eq:LookAtUnitary}
\hat{u}=(u_x, u_y, u_z)=(\sin{ \theta}\cos{\varphi},\sin{\theta}\sin{ \varphi},\cos{\theta})
\end{equation}

Sustituyendo la expresión (\ref{eq:LookAtUnitary}) dentro de los valores $x$ y $y$ quedan como:
\[x=r_x - \frac{r_z \sin{ \theta} \cos{\varphi}}{\cos{\theta}}\]
\[y=r_y - \frac{r_z \sin{\theta} \sin{ \varphi}}{\cos{\theta}}\]

Utilizando la identidad trigonométrica:
\[\tan{\theta} = \frac{\sin{\theta}}{\cos{\theta}}\]

Las ecuaciones para obtener la posición del objetivo siempre y cuando $z=0$ quedan:
\begin{equation}
\label{eq:xBidimentionalPosition}
x=r_x - r_z \tan{\theta}  \cos{\varphi}
\end{equation}

\begin{equation}
\label{eq:yBidimentionalPosition}
y=r_y - r_z \tan{\theta} \sin{\varphi}
\end{equation}

No obstante, el objeto a considerar no es un objeto bidimensional, es un balón con forma esférica que está sobre el plano del suelo, debido a esto se procede a hacer complementar las ecuaciones \ref{eq:xBidimentionalPosition} y \ref{eq:yBidimentionalPosition}, para obtener la posición (x,y,z) del centro del balón.

En la figura \ref{fig:ballProjection}(a) se observa un diagrama del balón en donde el vector \textit{look at} atravieza su centro e intersecta con el suelo en un punto en donde el balón no está. Esto representa un problema, ya que la posición en $x$ sufre una proyección, la cual incrementa mientras el balón tenga mayor radio. A esta proyección se le pondrá la variable $x_c$. Véase la figura \ref{fig:ballProjection}(b)


\begin{figure}
	\centering
	\includegraphics[scale=1.4]{images/ball_projection.pdf}
	\caption{(a) Bosquejo del vector \textit{look at} atravezando el centro del balón esférico; (b) Diagrama de la obtención de la distancia de proyección $x_c$.}
	\label{fig:ballProjection}
\end{figure}

Para obtener la magnitud de $x_c$ es necesario analizar el diagrama de la figura \ref{fig:ballProjection}(b), en donde $\theta$ es el ángulo \textit{pitch} y $\rho$ es el radio del balón. Así se puede obtener la siguiente igualdad:
\[\cot{\theta} = \frac{x_c}{\rho}\]\\

Despejando $x_c$
\[x_c = \rho \cot{\theta}\]

Debido a que $x_c$ se ve afectada también por la posición \textit{yaw} se deducen las siguientes igualdades para corregir el valor de la posición de $x$ y $y$ respectivamente:

\begin{equation}
\label{eq:xcForX}
x_{cx} = \rho \cot{\theta} \cos{\varphi}
\end{equation}

\begin{equation}
\label{eq:xcForY}
x_{cy} = \rho \cot{\theta} \sin{\varphi}
\end{equation}\\

De esta manera, se resta \ref{eq:xcForX}  a \ref{eq:xBidimentionalPosition} y \ref{eq:xcForY} a \ref{eq:yBidimentionalPosition}. Finalmente las ecuaciones resultantes son:

\[x = r_x - r_z \tan{\theta} \cos{\varphi} - \rho \cot{\theta} \cos{\varphi}\]
\[y = r_y - r_z \tan{\theta} \sin{\varphi} - \rho \cot{\theta} \sin{\varphi}\]
\[z = \rho \]


		
	\section{Estimación de estados mediante el Filtro de Kalman}
	Para poder impementar el Filtro de Kalman, se necesita primero tener un modelo matemático del sistema que se pretende tomar mediciones. Analizando el digrama de cuerpo libre de la figura \ref{fig:dynamic_model} se puede hacer un análisis dinámico del balón utilizando la segunda ley de Newton.

\begin{equation}
\sum F = m \ddot{x}
\label{eq:second_law}
\end{equation}

\begin{equation}
F-f_{fricción} = m \ddot{x}
\label{eq:equivalency_1}
\end{equation}

Cambiando el nombre de las variables a corde de nuestro diagrama y tomando en como fuerza de fricción dinámica $f_{fricción} = \mu_d m g $ la igualdad de fuerzas es:
\begin{equation}
F- \mu_d m g = m  \frac{\mathrm{d} \dot{x}}{\mathrm{d} t}
\label{eq:equivalency_2}
\end{equation}
	
Donde $\mu_d$ es el coeficiente de fricción dinámica, $m$ es la masa del balón y $g$ es la aceleración de la gravedad. Dado que ninguna fuerza mas que la de fricción es la que actúa sobre el balón, $F$ se iguala a cero, dejando el modelo el modelo cinemático de la expresión \ref{eq:mathematical_model_1}.

\begin{equation}
\frac{\mathrm{d} \dot{x}}{\mathrm{d} t} = - \mu_d g
\label{eq:mathematical_model_1}
\end{equation}

Despejando la variable $\dot{x}$ para se puede integrar la ecuación para obtener la velocidad en el tiempo $t$ del balón.
\begin{equation}
\int_{\dot{x}_0}^{\dot{x}} \mathrm{d} \dot{x} = -\mu_d g \int_{0}^{t} \mathrm{dt}  
\label{eq:mathematical_model_2}
\end{equation}

Resolviendo la ecuacion \ref{eq:mathematical_model_2} y despejando $\dot{x}$ se puede obtener la siguiente expresión:
\begin{equation}
\dot{x} = \dot{x}_{0} - \mu_d g t 
\label{eq:velocity_prediction}
\end{equation}

Una vez teniendo el modelo para predecir la velocidad del estado siguiente, se procede a integrar nuevamente para obtener su posición:
\begin{equation}
\int_{x_{0}}^{x} \mathrm{dx} = \int_{0}^{t} (\dot{x}_{0} - \mu_d g t) \mathrm{dt}
\label{eq:mathematical_model_4}
\end{equation}

Integrando y despejando $x$ se obtiene:
\begin{equation}
x = x_0 + \dot{x}_0 t - \frac{1}{2} \mu_d g t^2
\label{eq:position_prediction}
\end{equation}

%ESta ecuación es la solución de la ecuación diferencial que modela el movimiento
%De la eacuación 4.9 se tiene que:
Ya obtenida la ecuación para obtener la posición, despejamos la aceleración del modelo \ref{eq:equivalency_2}:
\begin{equation}
\ddot{x} = -\frac{1}{m}\mu_d m g + \frac{1}{m}F
\end{equation}

Pero F = 0. Es decir, que simplemente el balón comienza con $v_{0}$ diferente de cero y se va deteniendo.
Planteando esto en variables de estado:
\[ [x_1,x_2] = [x, \dot{x}]\]

En donde:
%\begin{eqnarray*}
\begin{equation}
\dot{x}_1 = x_2
\label{eq:state_variable_1}
\end{equation}

\begin{equation}
\dot{x}_2 = -\frac{1}{m}\mu_d m g + \frac{1}{m}F
\label{eq:state_variable_2}
\end{equation}

%\end{eqnarray*}

\begin{figure}
\centering
\includegraphics[scale=1.5]{images/dynamic_model.pdf}
\caption{Diagrama de cuerpo libre del objeto de interés}
\label{fig:dynamic_model}
\end{figure}

		\subsubsection*{Descripción del proceso de filtrado en el sistema}
Una vez que se obtuvo el modelo dinámico que describe el movimiento del balón, se procede a escribir las ecuaciones matriciales del Filtro de Kalman (Véase la sección 2.4). De este modo, este filtro es una serie de pasos que iterativamente se deben ir completando para estimar posiciones y velocidades de un objeto en movimiento (Figura: \ref{fig:kalman_extended_diagram}).

\begin{figure}
\centering
\includegraphics[scale=0.6]{images/kalman_extended_diagram.pdf}
\caption{Diagrama extendidio del funcionamiento paso a paso del Filtro de Kalman}
\label{fig:kalman_extended_diagram}
\end{figure}

		\subsubsection*{Modelado del sistema}
Para desarrollar las ecuaciones de extrapolación de estado se considerará que el balón se desplazará en el suelo sin elevarse o tener rebotes, por lo que el vector de estado $\hat{x}_n$ que describe la estimación de posición y velocidad en el plano \textit{(x, y)} es \ref{eq:state_vector}, siguiendo la nomenclatura que se vio en la sección 2.4.

\begin{equation}
\hat{x}_n = \begin{bmatrix}
{x_1}_n\\ 
{y_1}_n\\ 
{x_2}_n\\ 
{y_2}_n
\end{bmatrix}
\label{eq:state_vector}
\end{equation}

Dado que la única fuerza de entrada en el balón es la fricción, la cual depende del coeficiente de fricción dinámica y la gravedad el vector \textit{û} queda representado como:
\begin{equation}
\hat{u}_n = \begin{bmatrix}
F_x\\
F_y
\end{bmatrix} = 
\begin{bmatrix}
0\\
0
\end{bmatrix}
\label{eq:input_signal}
\end{equation}


Como se vio en la sección 2.4, la ecuación de extrapolación de estado es:
\begin{equation}
\hat{x}_{n+1,n} = F \hat{x}_{n,n} + G \hat{u}_{n,n} + \omega_n
\label{eq:extrapolation_equation}
\end{equation}

No obstante, como se vio en la deducción de la ecuación de posición, el modelo no es lineal, por lo que la ecuación \ref{eq:extrapolation_equation} se transforma a \ref{eq:efk} haciendo uso del Filtro de Kalman Extendido:

\begin{equation}
\hat{x}_{n+1,n} = f(\hat{x}_{n,n}, \hat{u}_{n,n}) + \omega_n
\label{eq:efk}
\end{equation}


\ref{eq:state_variable_2} es el modelo en variables de estado, continuo. Para el EKF se discretiza este modelo. Y la señal de entrada F = 0. 

El modelo discreto del sistema con ruido es:
\begin{eqnarray*}
x_{1_{n+1,n}} &=& x_{1_{n,n}} + \Delta t x_{2_{n,n}} + \omega_1\\ %Modelo1
y_{1_{n+1,n}} &=& y_{1_{n,n}} + \Delta t y_{2_{n,n}} + \omega_2\\ %Modelo2
x_{2_{n+1,n}} &=& x_{2_{n,n}} - \Delta t \frac{1}{m}\mu_d m g + \frac{1}{m} F_x + \omega_3\\ %Modelo3
y_{2_{n+1,n}} &=& y_{2_{n,n}} - \Delta t \frac{1}{m}\mu_d m g + \frac{1}{m} F_y + \omega_4 %Modelo4
\end{eqnarray*}
De estos cuatro estados, se miden dos:
\begin{eqnarray*}
z_1 &=& x_1 + v_1\\
z_2 &=& x_2 + v_2
\end{eqnarray*}
$\Omega = [\omega_1, ... \omega_4]$ es un vector de ruido con distribución normal, media cero, y matriz de covarianza $Q\in \mathbb{R}^{4\times 4}$.


La matriz de transición \textit{F} es:

\begin{equation}
F = \begin{bmatrix}
1 & 0 & \Delta t & 0\\ 
0 & 1 & 0 & \Delta t\\ 
0 & 0 & 1 & 0\\ 
0 & 0 & 0 & 1
\end{bmatrix}
\label{eq:transition_matrix}
\end{equation}


		\subsubsection*{Extrapolación de la incertidumbre de estimación}
	El proceso de estimación en el Filtro de Kalman es un modelo basado en la \textit{esperanza} de una serie de variables aleatorias para obtener el valor \textit{oculto} que se considera el real. Este proceso de filtrado considera que todos los errores tanto en medición como en estimación tienen una \textit{distribución gaussiana} por lo que cada incertidumbre tiene que expresarse como la varianza de una recompilación de datos.
	
\begin{equation}
P_{n+1,n} = F P_{n,n} F^{T} + Q
\label{eq:extrapolation_covariance}
\end{equation}

En donde:
\begin{equation}
Q =
\begin{bmatrix}
q_{x_1} & 0 & 0 & 0\\ 
0 & q_{y_1} & 0 & 0\\ 
0 & 0 & q_{x_2} & 0\\ 
0 & 0 & 0   & q_{y_2}
\end{bmatrix}
\end{equation}	

Dado que no se tiene estimación inicial, es conveniente que la varianza de estimación sea la matriz identidad, de este modo, en la primera corrección la ganancia de Kalman sería muy alta (cercana a 1) y tomaría como más confiables las mediciones R.

\begin{equation}
P_{0,0} =
\begin{bmatrix}
1 & 0 & 0 & 0\\ 
0 & 1 & 0 & 0\\ 
0 & 0 & 1 & 0\\ 
0 & 0 & 0 & 1
\end{bmatrix}
\end{equation}	

		\subsubsection*{Obtención de la ganancia de Kalman}
	Una vez teniendo la predicción del estado siguiente, es posible comparar las mediciones con las predicciones préviamente hechas para hacer un proceso de corrección. Para esto es necesario obtener el peso de la ganancia de Kalman que se describe con la siguiente ecuación:
\begin{equation}
K_n = P_{n,n-1} H^T (H P_{n,n-1}H^T + R_n)^{-1}	
\end{equation}
	En dónde H es la matriz de observación que sirve para asegurar que se pueda realizar la suma de la matriz de la varianza de la estimación con la del error en medición. En este caso la matriz es igual a:
	
\begin{equation}
H = 
\begin{bmatrix}
1 & 0 & 0 & 0\\ 
0 & 1 & 0 & 0\\ 
\end{bmatrix}
\end{equation}
	
y la matriz para el error en la medición es:

\begin{equation}
R_n = 
\begin{bmatrix}
r_x & 0 \\ 
0 & r_y \\ 
\end{bmatrix}
\label{matrix:measurement_error}
\end{equation}

		\subsubsection*{Corrección de estimación de estado}
	Con cada nueva medición, se procede a hacer la corrección del estado presente, haciendo uso del la ganancia del filtro para poder \textit{decidir} si es más fiable la medición o la predicción.

\begin{equation}
\hat{x}_{n,n} = \hat{x}_{n,n-1} + K_n(z_n - H \hat{x}_{n,n-1})
\end{equation}
	
		\subsubsection*{Corrección del error de covarianza}
	De una manera muy similar el error de covarianza se puede actualizar utilizando la ganancia de Kalman:

\begin{equation}
P_{n,n} = (I - K_n H) P_{n,n-1}
\end{equation}

	\section{Obtencion de los parametros}
		\subsection*{Coeficiente de fricción dinámica}
	De acuerdo con la ecuación \ref{eq:position_prediction} y como se vio a lo largo de su deducción, el modelo de la posición del balón respecto al tiempo, depende únicamente del coeficiente de fricción dinámica entre el balón y el material el cual se desplaza. Para obtener dicho coeficiente se optó por la solución más simple, la cual consiste en medir diréctamente la fuerza que se opone al desplazamiento, por medio de un dinamómetro de resorte marca \textit{Pasco} con rango de 0 a 10[N] y una resolución de 0.1[N]. Véase la figura \ref{fig:dynamometer}. 
	
\begin{figure}
\centering
\includegraphics[scale=0.4]{images/dynamometer.jpg}
\caption{Dynamómetro de resorte utilizado para obtener la fuerza de fricción}
\label{fig:dynamometer}
\end{figure}	

	Las medidas del dinamómetro del peso del balón dio $2.2[N]$, de la fuerza de fricción estática $F_{e} = 1.5[N]$ y de la fuerza de fricción dinámica $F_{d} = 0.8[N]$. 

	Por lo que se puede aplicar la segunda ley de Newton (de manera simplificada) para despejar el coeficiente de fricción deseado:
\begin{equation}
F_d = N\mu_d 
\end{equation}
	En donde $N$ es la fuerza normal a la superficie con la misma magnitud del peso del balón (por considerar que se está desplazando de manera horizontal). Por tanto la fuerza de fricción dinámica es:
	
\begin{equation}
	\mu_d = \frac{F_d}{N}
	      = 0.3636		
\end{equation}
	
		%\subsection*{Obtención de la incertidumbre en la estimacion}	
		\subsection*{Obtención de la incertidumbre en la medicion}
	De acuerdo con los datos de posición del balón (cuadro \ref{cuadro:x_position} y \ref{cuadro:y_position}) obtenidos por el sistema embebido, se puede obtener un promedio para el error de posición para recabar datos para la matriz de error de medición.

\begin{table}[]
\begin{tabular}{|c|c|c|}
\hline
\multicolumn{1}{|l|}{\textbf{Posición x real {[}m{]}}} & \multicolumn{1}{l|}{\textbf{Posición x obtenida {[}m{]}}} & \multicolumn{1}{l|}{\textbf{Error absoluto [m]}} \\ \hline
0.2                                                    & 0.290                                                     & 0.090                                        \\ \hline
0.3                                                    & 0.370                                                     & 0.070                                        \\ \hline
0.4                                                    & 0.489                                                     & 0.089                                        \\ \hline
0.5                                                    & 0.590                                                     & 0.090                                        \\ \hline
0.6                                                    & 0.745                                                     & 0.145                                        \\ \hline
0.7                                                    & 0.832                                                     & 0.132                                        \\ \hline
                                                       & Promedio:                                                 & 0.1026                                       \\ \hline
\end{tabular}
\caption{Error absoluto de la en posición \textit{x} del balón}
\label{cuadro:x_position}
\end{table}

\begin{table}[]
\begin{tabular}{|c|c|c|}
\hline
\multicolumn{1}{|l|}{\textbf{Posición y real {[}m{]}}} & \multicolumn{1}{l|}{\textbf{Posición y obtenida {[}m{]}}} & \multicolumn{1}{l|}{\textbf{Error absoluto [m]}} \\ \hline
-0.4                                                   & -0.39                                                     & 0.01                                         \\ \hline
-0.3                                                   & -0.32                                                     & 0.02                                         \\ \hline
-0.2                                                   & -0.22                                                     & 0.02                                         \\ \hline
-0.1                                                   & -0.071                                                    & 0.029                                        \\ \hline
0.1                                                    & 0.126                                                     & 0.026                                        \\ \hline
0.2                                                    & 0.183                                                     & 0.017                                        \\ \hline
0.3                                                    & 0.221                                                     & 0.079                                        \\ \hline
0.4                                                    & 0.37                                                      & 0.03                                         \\ \hline
                                                       & Promedio:                                                 & 0.0288                                       \\ \hline
\end{tabular}
\caption{Error absoluto de la posición en \textit{y} del balón}
\label{cuadro:y_position}
\end{table}

	 Obtenido el error de la medición en la posición del balón $e_x$ y $e_y$, se procede a obtener el error en la velocidad y dado que es obtenida indirectamente al dividir la diferencia de distancia entre la diferencia de tiempo, se deduce de la siguiente forma:	 

\begin{equation}
v_x = \frac{dx + de_x}{dt} 
    = \frac{dx}{dt} + \frac{de_x}{dt}
\end{equation}

\begin{equation}
v_y = \frac{dy + de_y}{dt}
    = \frac{dy}{dt} + \frac{de_y}{dt}
\end{equation}

	Por lo tanto, el error de medición en la velocidad de $e_{v_x}$ y $e_{v_y}$ es el cociente del error de distancia entre la diferencia de tiempo de muestreo. 

\begin{equation}
e_{v_x} = \frac{de_x}{dt}
		= \frac{0.1026}{0.0333}
		= 3.078 
\label{eq:ev_x}
\end{equation}

\begin{equation}
e_{v_y} = \frac{de_y}{dt}
		= \frac{0.0288}{0.0333}
		= 0.8648
\label{eq:ev_y}
\end{equation}

	De acuerdo con la matriz \ref{matrix:measurement_error} se pueden substituir los valores del promedio de los cuadros \ref{cuadro:x_position} y  \ref{cuadro:y_position} y los valores obtenidos de \ref{eq:ev_x} y \ref{eq:ev_y}:
	
\begin{equation}
R_n = 
\begin{bmatrix}
0.1026 & 0 & 0 & 0\\ 
0 & 0.0288 & 0 & 0\\ 
0 & 0 & 3.078 & 0\\ 
0 & 0 & 0 & 0.8648
\end{bmatrix}
\end{equation}

		%\subsection*{Obtención de la ganancia de Kalman}